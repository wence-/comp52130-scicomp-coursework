\documentclass[a4paper]{article}

\usepackage[margin=3cm]{geometry}
\usepackage{hyperref}
\usepackage{amsmath}
\usepackage{amssymb}
\usepackage{tikz}
\usetikzlibrary{matrix,shapes,positioning,fit,decorations.pathreplacing,arrows,calc}
\usepackage{graphicx}
\usepackage{cleveref}
\usepackage{tabularx}
\usepackage{booktabs}
\usepackage{subcaption}
 \usepackage[doi=true,
 url=false,
 sorting=ydnt,
 maxnames=10,
 backend=biber,
 isbn=false,
 giveninits=true,
 defernumbers=true]{biblatex}
\usepackage{minted}


\title{Performance analysis of a molecular dynamics application}

\author{Lawrence Mitchell\thanks{lawrence.mitchell@durham.ac.uk}}

\begin{document}
\maketitle

This assignment is to be completed and handed in via DUO. 
For deadlines, please consult DUO.

\section{Introduction}
\label{sec:introduction}

This coursework is to profile the performance of the \texttt{miniMD}
molecular dynamics MiniApp from the Mantevo project
(\url{https://www.mantevo.org}). Your goal is to determine the
computationally expensive parts of the code, and then perform more
detailed profiling using Likwid. Based on your measurements, you
should suggest some potential directions for obtaining more
performance in the code. You should perform your experiments on
Hamilton.

\section{Building the code}
\label{sec:build}

The code is provided as a number of C++ and header files, along with a
\texttt{Makefile} to build the executable. Obtain the code
(\texttt{code.tgz}) on DUO. After unpacking it with \texttt{tar xvf
  code.tgz}, you should see a directory \texttt{miniMD}.

The code can be built with either GCC or the Intel compiler. For GCC,
you need to do
\begin{minted}{sh}
$ module load gcc/9.3.0
\end{minted}
For Intel, you need to do
\begin{minted}{sh}
$ module load gcc/9.3.0
$ module load intel/2019.5
\end{minted}
To build after loading the appropriate modules, use \texttt{make}.
\begin{minted}{sh}
$ cd miniMD
$ make
\end{minted}

\section{Running the code}
\label{sec:run}
After building the code, you should have a \texttt{miniMD} executable.
To run the executable with default parameters just use
\texttt{./miniMD}, you should see output like the below
\begin{minted}{sh}
$ ./miniMD
# Create System:
# Done .... 
# miniMD-Reference 2.0 output ...
# Run Settings: 
        # Inputfile: in.lj.miniMD
        # Datafile: None

... Some details elided ...
# Performance Summary:
# nsteps natoms t_total t_force t_comm t_other performance
100 131072 6.136177 4.953426 0.094199 1.088552 2136053.094
\end{minted}
The final line shows a timing summary for different sections of the
code, along with performance in "number of atom updates per second"
(larger is better).

You can change the number of timesteps the program runs with
\texttt{./miniMD -n \#numsteps} (fewer steps runs for a shorter time).
The size of the problem (number of atoms) can be changed with
\texttt{./miniMD -s \#size}. The default is 32.
You can select between two algorithms for force computations with
\texttt{./miniMD --half\_neigh 0} and \texttt{./miniMD --half\_neigh 1}.

\subsection{Task specification}
\label{sec:task}
Your goal is to perform profiling of the provided code to identify
computational hotspots. Having identified the hotspots, you should
then explore them hotspots in more detail using likwid (by
appropriately annotating the identified hotspots). Recall that to use
likwid on Hamilton you need \texttt{module load likwid/5.0.1}.

You should explore the effect of different compiler flags (the default
set is not optimal) on the performance by editing \texttt{cflags.mk}
and rebuilding the executable with \texttt{make}.

In addition to compiler flags, the code implements two optimisations,
selected by setting \texttt{USE\_SIMD} and \texttt{USE\_PADDING} to
\texttt{Yes} in \texttt{cflags.mk}. What, if any, effect do these
settings have on the performance?


\subsection{Submission details}
\label{sec:submission}

You should write up your findings in a short report, detailing the
performance you achieved. The report should identify the hotspots that
you found and discuss, supported by appropriate evidence (either via
models or measurements) the best choice of compiler flags, and present
some ideas to improve performance further. Provide details of your
choices by including in the report the final version of the
\texttt{cflags.mk} file that resulted in the best performance.

In addition to this discussion, your report should present data and a
discussion answering the following questions:
\begin{enumerate}
\item What is the algorithmic complexity of the code for a \emph{fixed
    problem size} when you change the number of timesteps (via
  \texttt{-n})?
\item What is the algorithmic complexity of the code for a \emph{fixed
    number of timesteps} when you increase the number of atoms (via
  \texttt{-s})?
\item What can you say about these results when you change the
  algorithm used for force computations (\texttt{--half\_neigh})?
\end{enumerate}

You should submit
\begin{itemize}
\item \textbf{Report (max 3 pages)}.

  \textbf{This document \emph{must} be submitted as PDF}.
\end{itemize}

\section{Marks}
\label{sec:marks}

There are in total 100 marks for the coursework, they
will be awarded as follows:

\begin{center}
  \renewcommand\tabularxcolumn[1]{m{#1}}
  \begin{tabularx}{0.9\linewidth}{Xc}
    \toprule
    Descriptor                                                  & Marks \\
    \midrule
    Discussion of algorithmic complexity                        & 30    \\
    Identification of code hotspots                             & 20    \\
    Appropriate use of performance measurements                 & 20    \\
    Appropriate use of performance model(s)                     & 10    \\
    \midrule
    Quality of writing and presentation of data                 & 20    \\
    \midrule
    Total                                                       & 100   \\
    \bottomrule
  \end{tabularx}
\end{center}

\end{document}
