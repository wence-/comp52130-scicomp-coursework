\documentclass[a4paper]{article}

\usepackage[margin=3cm]{geometry}
\usepackage{hyperref}
\usepackage{amsmath}
\usepackage{amssymb}
\usepackage{tikz}
\usetikzlibrary{matrix,shapes,positioning,fit,decorations.pathreplacing,arrows,calc}
\usepackage{graphicx}
\usepackage{cleveref}
\usepackage{tabularx}
\usepackage{booktabs}
\usepackage{subcaption}
 \usepackage[doi=true,
 url=false,
 sorting=ydnt,
 maxnames=10,
 backend=biber,
 isbn=false,
 giveninits=true,
 defernumbers=true]{biblatex}
\usepackage{minted}


\bibliography{papers.bib}


\title{Performance analysis and optimisation of matrix-matrix multiplication}

\author{Lawrence Mitchell\thanks{lawrence.mitchell@durham.ac.uk}}

\begin{document}
\maketitle

This assignment is to be completed and handed in via DUO. 
For deadlines, please consult DUO and the level handbook.

\section{Introduction}
\label{sec:introduction}
Matrix-matrix multiplication is at the core of much of scientific
computing. In this coursework you will study, and improve the
performance of, this core computational kernel for the case of dense
matrix-matrix multiplication. You are provided with a template code
that implements the operation
\begin{equation}
  \label{eq:1}
  C = C + A B
\end{equation}
where $C \in \mathbb{R}^{m \times n}$, $A \in \mathbb{R}^{m \times k}$,
and $B \in \mathbb{R}^{k \times n}$ are rectangular matrices.

\subsection{The BLIS approach to matrix-matrix multiplication}
\label{sec:blis}

The provided code implements the loop blocking scheme developed by
Kazushigo Goto, Robert van der Geijn and co-workers in the BLIS
(BLAS-like Library Instantiation Software) framework. It is described
in detail two papers: \textcite{Goto:2008} and \textcite{Zee:2015}.
You do not have to read these to make progress on the coursework, they
may be interesting and/or useful when deciding how to choose
parameters. The na{\"i}ve implementation of matrix-matrix
multiplication uses three nested for-loops.
\begin{minted}{c}
for (j = 0; j < n; j++)
  for (p = 0; p < k; p++)
    for (i = 0; i < m; i++)
      c[i, j] = c[i, j] + a[i, p] * b[p, j];
\end{minted}
This implementation of the algorithm, however, exhibits poor spatial
and temporal locality, and is not amenable to vectorisation. The BLIS
approach performs a particular loop-tiling that effectively blocks for
all levels of the memory hierarchy. The innermost loop is a
``microkernel'' that must be tuned to each architecture in turn.


\subsection{Building the code}
\label{sec:build}

The code is provided as a number of C and header files, along with a
\texttt{Makefile} to build the executable. After unpacking the code,
you should see the following files
\begin{minted}{sh}
# unpack code
$ tar zxvf code.tgz
$ ls code
cflags.mk
gemm.c
Makefile
micro-kernel.c
optimised-gemm.c
parameters.h
\end{minted}

To build the executable, type \texttt{make}.
\begin{minted}{sh}
$ cd code
$ make
\end{minted}
This creates an executable \texttt{gemm}. Running it provides guidance
on usage
\begin{minted}{sh}
$ ./gemm
Invalid arguments.
Usage: ./gemm M N K mode
Where M, N, and K are the dimensions of the problem.
'mode' is one of BENCH or CHECK
\end{minted}
Running the code with arguments for the matrix sizes and the mode
\texttt{BENCH} produces a line of output with six columns
\begin{minted}{sh}
$ ./gemm 128 2048 997 BENCH
128 2048 997 0.0608405 5.22715e+08 8.59156e+09
\end{minted}
The columns are, in order
\begin{minted}{sh}
M N K Time(s) FLOP FLOP/s
\end{minted}
where \texttt{FLOP} is the total number of floating point operations
required for the multiplication and \texttt{FLOP/s} is the floating
point throughput in \texttt{FLOP} per second (also available by
dividing the \texttt{FLOP} column by the time column.

\subsection{Task specification}
\label{sec:task}
The provided implementation implements the BLIS loop structure
correctly, but does not have optimal values for the loop blocking
parameters. There are five parameters to tune:
\begin{enumerate}
\item \texttt{MR} and \texttt{NR} control the size of the inner-most
  micro-kernel loops. These should be chosen quite small, so that the
  computation fits in registers.
\item \texttt{MC}, \texttt{KC}, and \texttt{NC} control the sizes of
  the blocks that are streamed from main memory into levels of the
  cache. The BLIS papers provide some guidance on how to choose these
  appropriately.
\end{enumerate}
To change these, edit the file \texttt{parameters.h} and rebuild the
executable with \texttt{make} (don't forget this second step!).

Your goal is to choose a set of parameters that achieves good
performance for this algorithm over a range of matrix sizes
(controlled by the \texttt{M}, \texttt{N}, and \texttt{K} arguments to
the executable). Background information on how to choose
parameters (and bounds on memory movement) is available in
\textcite{Low:2016} and \textcite{Smith:2017}.

In addition to changing parameters, you might also (depending on the
compiler you select) need to add some vectorisation and loop unrolling
pragmas to the innermost micro-kernel. To do this, edit the file
\texttt{micro-kernel.c} and rebuild the executable with \texttt{make}.

Finally, you can also experiment with different compilers (if
available) and compiler flags (the default set is not optimal) by
editing \texttt{cflags.mk} and rebuilding the executable with
\texttt{make}. If you set \texttt{USE\_LIKWID = Yes} in
\texttt{cflags.mk} then the code will be compiled with likwid marker
instrumentation. You can then use \texttt{likwid-perfctr} and the
marker API to carry out performance measurements of the interesting
parts of the code.

You should not need to edit any of the other files. This coursework is
about single-core, vectorised optimisation, rather than shared memory
(OpenMP) or distributed memory (MPI) optimisation. Hence you do not
need to try and implemented a shared- or distributed-memory
parallelisation strategy.

\pagebreak
\subsection{Submission details}
\label{sec:submission}

You should write up your findings in a short report, detailing the
performance you achieved. The report should discuss, supported by
appropriate evidence (either via models or measurements) the reasons
for your choices of parameters, compiler flags, and code
annotations/optimisations. You should additionally discuss what else
you think would be required to improve performance (or else state why
you think you have achieved the maximum performance possible) with
reference to appropriate models and measurements.

You should submit

\begin{itemize}
\item \textbf{Report (max 3 pages)}.

  \textbf{This document \emph{must} be submitted as PDF}.

  
\item A zip file containing the three modified files with the optimal set of
  parameters/compiler flags/etc\dots
  \begin{itemize}
  \item \texttt{cflags.mk}
  \item \texttt{micro-kernel.c}
  \item \texttt{parameters.h}
  \end{itemize}
\end{itemize}

\section{Marks}
\label{sec:marks}

There are in total 100 marks for the coursework, they
will be awarded as follows:

\begin{center}
  \renewcommand\tabularxcolumn[1]{m{#1}}
  \begin{tabularx}{0.9\linewidth}{Xc}
    \toprule
    Descriptor                                                  & Marks \\
    \midrule
    Appropriate use of performance measurements                 & 20    \\
    Appropriate use of performance model(s)                     & 20    \\
    Quality of writing                                          & 15    \\
    Quality of presentation of data                             & 15    \\
    \midrule
    Performance of submitted code                               & 30    \\
    \midrule
    Total                                                       & 100   \\
    \bottomrule
  \end{tabularx}
\end{center}

\printbibliography

\end{document}
